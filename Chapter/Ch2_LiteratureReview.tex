Recent literature shows that various approaches can be used to control soft robots. This chapter will touch upon these control techniques. The work of Thuruthel et al.(2018) \cite{george2018control} presents a good overview of recent findings in soft robotic control for pneumatic actuated and tendon-driven soft robots. In this work, three control approaches are distinguished, namely, model-free, hybrid and model-based control. A model free control strategy does not use any kind of kinematic or dynamical description in its control approach. Instead, it exploits learning-based algorithms to dynamically control the robotic system. Hybrid controllers combine learning or empirical elements with a model-based approach. The model-based controllers rely on a analytical description of robotic system. Furthermore, for each control approach, a sub-division between kinematic and dynamical control is made. Here, kinematic control is defined as a zero order input-output relation. A change in input is directly observed in output, therefore this approach is lower-level compared to dynamical control. For the latter, the configuration space and/or task space variables' velocities are used in the control algorithm. 


The research conducted on the Bionic Handling Assistant (BHA), as shown in Figure \ref{fig:BHA}, shows that there is no unique approach to control the BHA. The theory presented \cite{rolf2013efficient}, used a model-free control approach. Here the technique of goal babbling is used to the the BHA, where goal babbling is defined as the ``bootstrapping of a coordination skill by repetitively trying to accomplish multiple goals related to that skill" \cite{rolf2012goal}. The goal of this control approach is to learn an inverse kinematic model, instead of feeding an analytically obtained inverse model directly to the algorithm. Later, a hybrid controller for the BHA was proposed in \cite{reinhart2017hybrid}. In this approach, an inverse kinematic model was used together with a learning model. Previous to this hybrid controller, a model-based controller was implemented on the BHA \cite{mahl2014bhakin}. However, this model-based controller only used a kinematic description of the soft robot. Later, a dynamic controller for the BHA was designed by \cite{falkenhahn2016dynamic}. In its attempt to control the system, a nonlinear length controller based on feedback linearization was used, together with linear PD and feedforward control. 

In \cite{wang2013visual} visual servo control is used to control a cable-driven soft robotic manipulator. The manipulator, inspired by an octopus' tentacle, is shaped like a cone. On the outer surface of the soft robot four cables are uniformly distributed. These cables are on one end connected to the tip of the manipulator and on the other side connected to pulleys. By winding and unwinding the cables on the pulleys, the end-effector can be positioned in the task space. A kinematic-based visual servo controller is designed, based on linear PD feedback control with gravity compensation. Position sensing is done by a camera mounted at the tip of the robot, and a fixed feature point. Based on this visual information, a jacobian matrix is estimated, which is used in the kinematic controller.

In \cite{zhang2017visual} also visual servo control was implemented to control a tendon-driven soft robotic manipulator. Again, kinematic control is used for a reference tracking problem. Contrary to the previous research, where a jacobian matrix is estimated, here the jacobian is calculated by running a real-time FEM model. A control architecture is set-up in which the reference input is ran through the FEM model, which outputs a jacobian and expected end-effector position. The jacobian is used in the controller to control the actual manipulator. The FEM model's expected end-effector position and actual end-effector position are used in a second controller to enhance jacobian estimation in the FEM model.

A model-based controller that uses a dynamical model in its control loop is presented in \cite{della2020model}. The robots infinite dimensionality is resolved by regarding the robot being composed of six individual actuatable segments. A Piecewise Constant Curvature (PCC) approach was used to describe the kinematics of the robotic actuator. For each segment Denavit-Hartenberg (DH) is used to describe the rigid robot equivalent, which is then used to derive a dynamic model. The proposed controller essentially is a computed torque controller with an additional PD controller. 

In aforementioned work, \cite{della2020model}, the infinite dimensionality of the soft robot is solved by assuming the robot consists of a fixed amount of segments. The robot that we will be developing a controller for theoretically has infinite degrees of freedom too, making it highly under-actuated. In the developed dynamic model \cite{caasenbrood2020} the degrees of freedom are limited by approximating strains with shape functions. The more shape functions used to describe the strain in the robot, the higher accuracy is achieved. However, the physical system will always be under-actuated. Therefore, control theories applied to under-actuated (soft) robots need to be applied in order to develop a model-based controller.

In order to design a model-based controller to the under-actuated soft robot, theory presented in \cite{della2019exact} might be of use. This paper provides a theory that allows conversion of specific controllers designed for fully actuated systems, to be used in reduced input cases. To apply this theory, mappings from the task space to the \textit{synergy space} need to be determined. These mappings allows us to use the developed dynamical in the control loop to a higher extend, as more degrees of freedom can be taken into account than strictly can be actuated. 

Inspiration for the model-based controller that we aim to develop is drawn from \cite{spong1996energy}. In this paper an energy based control strategy is presented, applied to classic under-actuated mechanical systems. Our aim is to apply theory that was originally developed for classic robotic control to the field of soft robotics.


%Sliding mode control (SMC) can be used to control under-actuated systems. SMC can be an effective tool in controller design, as it is insensitive to model errors and external disturbances. The cause of this phenomenon is that the behaviour on the sliding mode depends only on the defined switching surface, not the structural properties of the system \cite{liu2013survey}. In \cite{xu2008sliding}, a theory is presented to globally stabilize all degrees of freedom, including the non-linearly coupled, indirectly actuated degrees of freedom for a system in cascaded form. By defining a sliding surface, the un-actuated degrees of freedom can be stabilized, and the actuated degrees of freedom controlled.

%In \cite{liu2013survey}, the theory of fuzzy control for under-actuated systems is presented. This control strategy is used to mimic human logic and can therefore deal with imprecise, uncertain or qualitative decision-making situations. This control strategy is used to control mechanical systems for which model making is difficult \cite{michels2007fuzzy}, plenty of model-based fuzzy controllers have been designed, e.g. \cite{begovich2002takagi} and \cite{tao2008design}. 

%Instead of SMC and fuzzy control, backstepping \cite{khalil2002nonlinear} can be used to control the system. This control approach transforms the system to a new recursive form, where a sequence of "virtual" systems is created. By selecting the "virtual" input, the system can be made globally asymptotically stable. Backstepping has been previously used for under-actuated systems such as in \cite{madani2006backstepping}.

%In \cite{spong1994linear}, non-linear partial feedback linearization is discussed. The mechanical system is described in generalized coordinates, and divided in actuated and un-actuated degrees of freedom. The theory presented, shows that under "strong inertial coupling" the actuated joints can linearize the un-actuated degrees of freedom. 

%Energy based control \cite{spong1996energy} is another approach to control the under-actuated mechanical system. In essence, nonlinear partial feedback linearization is used, together with energy shaping. For the system an energy function is determined, enabling to calculate the total energy at a given point. By using multiple controllers and switching between those, under-actuated systems can stabilize around their desired end-effector position. The theory and application of energy based control for Euler-Lagrange systems has been captured very well in \cite{ortega2013passivity}.
